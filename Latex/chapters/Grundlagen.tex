% !TEX root = ../VPJ.tex

\chapter{Grundlagen und Konzepte}
\label{sec:Grundlagen}

Grundlagen

\section{Qt}

\subsection{Eventbasiertes System}
\label{sec:Eventbasiert}

\subsection{Sockets}
\label{sec:QTSocket}
https://doc.qt.io/qt-5/qudpsocket.html
\cite{qt_socket}

\subsection{Datenbank in Qt}
\label{sec:QTDatabase}
%    C:\Program Files\MySQL\MySQL Connector.C 6.1\lib
MySQLLib.dll

https://doc.qt.io/qt-5/qsqldatabase.html
\cite{qt_database}

\subsection{State Machines in Qt}
\label{sec:StateMachines}
https://doc.qt.io/qt-5/statemachine-api.html
\cite{qt_statemachine}
\inlinetodo{erwaehnen, dass nur entered() benutzt wird}

\subsection{QListWidget}
\label{sec:QListWidgetItem}


Das QListWidget ermöglicht es, selbst programmierte QListWidgetItems in einer Liste zu organisieren und darzustellen. Zusätzlich werden diverse Funktionen bereitgestellt, mit denen sich die QListWidgetItems ordnen, ergänzen oder finden lassen (vgl. \cite{qt_listwidget}). 

Über die Auswahl der allgemeinen Widgets kann das QListWidget im Layout-Manager ausgewählt werden. 
