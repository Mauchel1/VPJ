\chapter{Fertigungsrechner}\label{kap:Fertigungsrechner}
 Der Fertigungsrechner ist Teil des Gewerk 1 und besteht hauptsächlich aus der MySQL-Datenbank und der Fertigungsüberwachung. Die MySQL-Daten\-bank be\-in\-hal\-tet alle relevanten Daten des Fertigungsprozesses. Der Entwurf und die Implementierung der Datenbank wird in Kapitel \ref{kap:Datenbank} beschrieben. Über das lokale LAN-Netzwerk kann der Rechner der Fertigungsplanung auf den Fertigungsrechner beziehungsweise die hier abgelegte MySQL-Datenbank zugreifen. 
 Die Fertigungsüberwachung steuert die Ansteuerung und Auswertung der RFID-Schreib-Lese-Köpfe und legt die Daten der Werkstückverfolgung in der Datenbank ab. Mit den Schreib-Lese-Köpfen ist eine Verfolgung der Werkstücke im Fertigungsprozess möglich, da jeder An- und Abmeldevorgang an den Fertigungsstationen 2-7 und dem Rohteil- so wie dem Fertigteillager überwacht und gespeichert wird.  Über ein Interface und das Bussystem Profibus-DP sind die Schreib-Lese-Köpfe an den Fertigungsrechner angeschlossen. Die Fertigungsüberwachung wird in Kapitel \ref{kap:Fertigungsueberwachung} beschrieben.  In der Abbildung \ref{fig:Fertigungsrechner} ist eine Übersicht über die Komponenten des Fertigungsrechners und die Schnittstellen dargestellt.
 \begin{figure}[hbt]
	    \centering
	    \includegraphics[width=1\linewidth]{Bilder/Fertigungsrechner.png}
        \caption{Fertigungsrechner und Schnittstellen}
        \label{fig:Fertigungsrechner}
\end{figure}
 